4_OSINT_MINYANDA_TEX
\documentclass[12pt,a4paper]{report}
\usepackage[utf8]{inputenc}
\usepackage[T1]{fontenc}
\usepackage[french]{babel}
\usepackage{geometry}
\geometry{margin=2.5cm}
\usepackage{graphicx}
\usepackage{array}
\usepackage{longtable}
\usepackage{hyperref}
\usepackage{titlesec}
\titleformat{\chapter}[hang]{\bfseries\huge}{\thechapter.}{1em}{}
\titleformat{\section}[hang]{\bfseries\Large}{\thesection.}{1em}{}
\titleformat{\subsection}[hang]{\bfseries\large}{\thesubsection.}{1em}{}

\begin{document}

\begin{titlepage}
    \centering
    \textbf{\LARGE École Nationale Supérieure Polytechnique de Yaoundé} \\[1cm]
    \textbf{\large Département : HN – Cybersécurité et Investigations Numériques} \\[1.5cm]
    \Huge \textbf{Rapport d’investigation numérique sur Embolo Mvogo Shawn Douglas} \\[1.5cm]
    \Large \textbf{Sujet :} Analyse de l’empreinte numérique de M. Embolo Mvogo Shawn Douglas \\[1cm]
    \textbf{Réalisé par :} José Loïc Minyanda (Matricule : 22P071) \\[0.3cm]
    \textbf{Binôme :} Embolo Mvogo Shawn Douglas (Matricule : 22P072) \\[0.3cm]
    \textbf{Encadrant :} Mr. MINKA MI NGUIDJOI Thierry Emmanuel \\[0.5cm]
    \textbf{Année académique :} 2025–2026 \\[2cm]
    \vfill
\end{titlepage}

\tableofcontents
\newpage

\chapter{Introduction générale}
Dans le cadre du cours d’investigation numérique (OSINT), il est demandé à chaque étudiant d’effectuer une analyse d’empreinte numérique de son binôme à partir des informations accessibles publiquement. L’objectif de ce travail est de démontrer la capacité de l’investigateur à collecter, analyser et vérifier des informations en ligne sans enfreindre la vie privée de la personne concernée.

Ce rapport présente les résultats de l’investigation menée sur M. Embolo Mvogo Shawn Douglas, étudiant à l’École Nationale Supérieure Polytechnique de Yaoundé, en spécialité HN-CIN4. Le travail suit une méthodologie rigoureuse de collecte et d’analyse des données ouvertes, tout en respectant les principes d’éthique numérique.

\chapter{Informations initiales connues}
Avant d’entamer l’investigation, les informations suivantes étaient déjà connues ou directement accessibles :

\begin{longtable}{|p{6cm}|p{8cm}|}
\hline
\textbf{Élément} & \textbf{Détail} \\
\hline
Nom complet & Embolo Mvogo Shawn Douglas \\
\hline
Matricule & 22P072 \\
\hline
Spécialité & HN-CIN4 (Cybersécurité et Investigations Numériques) \\
\hline
Date de naissance & 17 décembre 2002 \\
\hline
Formation antérieure & Baccalauréat au Lycée Bilingue de Yaoundé \\
\hline
Parcours universitaire & Faculté des Sciences, Université de Yaoundé I (2021) \\
\hline
Admission à l’ENSPY & Par concours en 2022 \\
\hline
Réseaux sociaux connus & Facebook, LinkedIn, Instagram \\
\hline
Comptes identifiés & - Shawn Douglas Embolo Mvogo (Facebook, LinkedIn, Instagram) \\ - Shawn Douglas (Facebook) \\
\hline
\end{longtable}

Ces éléments constituaient la base de départ pour la recherche et l’exploration en ligne.

\chapter{Méthodologie d’investigation}
\section{Approche générale}
La méthodologie adoptée repose sur une démarche OSINT (Open Source Intelligence). Elle consiste à rechercher, collecter et analyser les informations publiquement accessibles sur Internet, tout en respectant les cadres éthiques et légaux.

Les principales étapes ont été :
\begin{enumerate}
    \item Définition des mots-clés et variantes de noms utilisés pour la recherche.
    \item Exploration manuelle sur les moteurs de recherche (Google, Bing, DuckDuckGo).
    \item Analyse des réseaux sociaux : Facebook, LinkedIn, Instagram.
    \item Recoupement et validation des informations trouvées.
    \item Évaluation du niveau d’exposition et de cohérence numérique.
\end{enumerate}

\section{Outils et techniques OSINT utilisés}
\begin{longtable}{|p{4cm}|p{5cm}|p{5cm}|}
\hline
\textbf{Catégorie} & \textbf{Outils/Techniques utilisés} & \textbf{Objectif} \\
\hline
Moteurs de recherche & Google, Bing & Retrouver les traces numériques du nom \\
\hline
Réseaux sociaux & Facebook, LinkedIn, Instagram & Identifier les profils et publications \\
\hline
Validation & Recoupement manuel & Vérifier la cohérence entre les données \\
\hline
Documentation & Captures d’écran, notes & Archivage des résultats \\
\hline
\end{longtable}

\section{Considérations éthiques}
Aucune tentative d’accès à des données privées ou protégées n’a été effectuée. Toutes les informations ont été obtenues à partir de sources publiques, conformément aux bonnes pratiques de l’investigation numérique académique.

\chapter{Résultats obtenus}
\section{Facebook}
Deux comptes distincts ont été retrouvés :
\begin{itemize}
    \item \textbf{Compte 1 :} Shawn Douglas Embolo Mvogo
    \begin{itemize}
        \item Photo de profil amateur.
        \item Activité modérée : premier compte inactif depuis 2017.
        \item Réseau d’amis majoritairement composé d’anciens camarades du Lycée Bilingue et de membres de la famille.
        \item Utilisation d’un ton enfantin.
    \end{itemize}
    \item \textbf{Compte 2 :} Shawn Douglas
    \begin{itemize}
        \item Photo plus décontractée.
        \item Publications plus récentes, centrées sur la vie d’adolescent et les loisirs.
        \item Aucun lien probable avec son premier compte.
    \end{itemize}
\end{itemize}

\section{LinkedIn}
\begin{itemize}
    \item Compte : Shawn Douglas Embolo Mvogo
    \item Informations :
    \begin{itemize}
        \item Étudiant à l’ENSPY.
        \item Réseau professionnel restreint mais cohérent.
        \item Profil rédigé uniquement en français.
    \end{itemize}
\end{itemize}

\section{Instagram}
\begin{itemize}
    \item Compte : Shawn Douglas Embolo Mvogo
    \item Faible activité.
    \item Contenu presque inexistant.
    \item Peu de publications personnelles sensibles, indiquant une certaine prudence numérique.
\end{itemize}

\section{Autres traces numériques}
\begin{itemize}
    \item Aucune mention significative sur GitHub, ResearchGate, Google Scholar ou autres plateformes professionnelles.
    \item Pas de traces sur les bases de données WHOIS ou sur des forums publics.
\end{itemize}

\chapter{Analyse et comparaison}
\begin{longtable}{|p{3cm}|p{5cm}|p{5cm}|p{3cm}|}
\hline
\textbf{Aspect} & \textbf{Informations connues} & \textbf{Informations trouvées} & \textbf{Commentaire} \\
\hline
Identité & Étudiant à l’ENSPY, 22P072 & Confirmé sur LinkedIn et Facebook & Correspondance parfaite \\
\hline
Formation & Parcours via Université de Yaoundé I & Confirmé via publications et échanges & Cohérence totale \\
\hline
Réseaux sociaux & Facebook, LinkedIn, Instagram & Comptes réels retrouvés & Cohérents \\
\hline
Comportement numérique & Réservé & Publications sobres et neutres & Excellente maîtrise \\
\hline
Risques d’exposition & Faible & Pas d’informations sensibles & Bonne hygiène numérique \\
\hline
\end{longtable}

\section{Observation générale}
L’image numérique de M. Embolo Mvogo Shawn Douglas est cohérente, maîtrisée et conforme à son profil d’étudiant en cybersécurité. Il sépare bien ses sphères personnelles et professionnelles, ce qui réduit les risques d’exposition numérique.

\chapter{Discussion}
Cette investigation révèle un étudiant conscient des enjeux de la confidentialité et du contrôle de son identité numérique. L’utilisation professionnelle de LinkedIn et la retenue observée sur Facebook témoignent d’une bonne maturité numérique. Peu de données personnelles (numéro, adresse, informations familiales) sont accessibles, signe d’une culture de la discrétion.

Toutefois, la présence de deux comptes Facebook pourrait parfois prêter à confusion pour un recruteur ou un chercheur d’informations, d’où l’importance d’une gestion centralisée de son identité en ligne.

\chapter{Conclusion}
L’investigation numérique menée sur M. Embolo Mvogo Shawn Douglas a permis de confirmer la majorité des informations initialement connues. Son empreinte numérique est limitée mais structurée, avec une nette distinction entre les espaces personnels et professionnels. Ce travail démontre non seulement la méthodologie OSINT appliquée, mais aussi l’importance d’une hygiène numérique responsable dans le monde académique et professionnel.

\textbf{Recommandations :}
\begin{itemize}
    \item Harmoniser les comptes Facebook pour plus de clarté.
    \item Développer son profil LinkedIn avec des expériences, projets et compétences.
    \item Continuer à préserver ses données personnelles.
\end{itemize}

\chapter{Bibliographie et annexes}
\textbf{Sources consultées :}
\begin{itemize}
    \item Google (recherche « Embolo Mvogo Shawn Douglas »)
    \item Facebook (profils : Shawn Douglas, Shawn Douglas Embolo Mvogo)
    \item LinkedIn (Shawn Douglas Embolo Mvogo)
    \item Instagram (Shawn Douglas Embolo Mvogo)
    \item Documentation de cours OSINT – ENSPY 2025
\end{itemize}

\textbf{Annexes :}
\begin{itemize}
    \item Captures d’écran anonymisées des profils.
    \item Tableau récapitulatif des sources et liens.
\end{itemize}

\end{document}
