MINYANDA_RESUME_0 
\documentclass[12pt,a4paper]{article}
\usepackage[utf8]{inputenc}
\usepackage[T1]{fontenc}
\usepackage[french]{babel}
\usepackage{lmodern}
\usepackage{hyperref}
\usepackage{enumitem}

\title{Résumé du cours : \\ \textbf{Théories et Pratiques de l’Investigation Numérique}}
\author{MINYANDA YONGUI JOSE LOIC \\ 22P071 \\ CIN4}
\date{}

\begin{document}

\maketitle

\section*{Introduction}

Ce résumé présente une synthèse structurée du manuel \textit{Théories et Pratiques de l’Investigation Numérique} de Thierry Emmanuel Minka Mi Nguidjoi. 
Il met en évidence les fondements philosophiques et éthiques de la discipline, son évolution historique, ses modèles théoriques et normes internationales, ainsi que ses méthodes pratiques, outils et enjeux face à l’informatique post-quantique. 
Enfin, il illustre ces concepts par un cas pratique appliqué au Cameroun.

\section{Fondements philosophiques et éthiques}

L’investigation numérique n’est pas seulement un ensemble d’outils techniques : c’est une discipline qui engage la responsabilité de l’investigateur face à la société. 
L’ouvrage insiste sur la notion de \textit{philosophe-praticien}, capable d’utiliser ses compétences dans le respect de principes éthiques.

Le Trilemme CRO (Confidentialité, Fiabilité, Opposabilité) structure la réflexion autour de la preuve numérique :

\begin{itemize}
    \item \textbf{Confidentialité} : protéger les données sensibles, respecter la vie privée.
    \item \textbf{Fiabilité} : assurer la reproductibilité et l’intégrité des résultats.
    \item \textbf{Opposabilité} : garantir que les preuves résistent devant un tribunal.
\end{itemize}

Une charte déontologique formalise cet engagement, à travers dix commandements tels que : ne pas causer de dommages aux systèmes, documenter ses méthodes, protéger la chaîne de custody et témoigner avec honnêteté. 
Quatre piliers éthiques complètent cette charte : intégrité, proportionnalité, responsabilité et service.

Ainsi, la technique doit toujours être guidée par la sagesse et le sens du devoir, car l’investigation numérique impacte directement des vies humaines et des droits fondamentaux.

\section{Historique et évolution de la discipline}

La discipline a émergé dans les années 1970 avec les premiers litiges liés à l’informatique. Son évolution peut être découpée en grandes phases :

\begin{itemize}
    \item 1970--1990 : \textbf{Les prémices}. Apparition des premiers cas judiciaires liés aux ordinateurs, comme l’affaire du groupe « 414s ».
    \item 1990--2000 : \textbf{Professionnalisation}. Création d’unités spécialisées, enquêtes emblématiques comme l’Operation Sundevil ou l’arrestation de Kevin Mitnick.
    \item 2000--2010 : \textbf{Standardisation}. Développement de cadres internationaux (ISO, NIST), avec des affaires marquantes comme Enron.
    \item 2010--2020 : \textbf{Big data et Cloud}. Émergence des cyberattaques massives (Silk Road, Panama Papers, WannaCry).
    \item 2020 à nos jours : \textbf{IA et post-quantique}. Attaques sophistiquées comme SolarWinds, posant la question de l’avenir de la cryptographie.
\end{itemize}

Ces étapes montrent que l’investigation numérique évolue constamment en fonction des menaces, des technologies et des besoins juridiques.

\section{Cadre théorique et normes}

\subsection{Modèles théoriques}

\begin{itemize}
    \item DFRWS (2001) : définit les étapes de collecte, préservation, analyse et présentation.
    \item Casey (2004) : met l’accent sur le contexte judiciaire et l’admissibilité des preuves.
    \item ISO/IEC 27037 (2012) : norme internationale encadrant l’identification, la collecte et la préservation des preuves numériques.
\end{itemize}

\subsection{Normes et standards internationaux}

\begin{itemize}
    \item NIST SP 800-86 (USA) : guide technique détaillé pour la gestion des enquêtes numériques.
    \item RFC 3227 : introduit la notion d’« ordre de volatilité ».
    \item ACPO Good Practice Guide (Royaume-Uni) : quatre principes garantissant l’intégrité des preuves.
    \item Normes émergentes : Cloud Forensics, IoT Forensics.
\end{itemize}

Ces cadres favorisent la coopération entre pays et assurent que les preuves recueillies soient reconnues au niveau international.

\section{Méthodes, outils et anti-forensique}

\subsection{Méthodologies d’investigation}

\begin{itemize}
    \item SANS FOR508 : centré sur la réponse aux incidents.
    \item CERT/CC : processus d’investigation lié à la cybersécurité opérationnelle.
    \item ENISA Framework : méthodologie européenne adaptée aux contextes réglementaires stricts.
\end{itemize}

\subsection{Outils de l’investigateur moderne}

\begin{itemize}
    \item Acquisition \& imagerie : FTK Imager, EnCase.
    \item Analyse mémoire : Volatility.
    \item Timeline analysis : Plaso, Autopsy.
    \item SIEM et logs : analyse centralisée de grandes volumétries.
    \item Intelligence artificielle : machine learning et deep learning.
\end{itemize}

\subsection{Anti-forensique et contremesures}

Les criminels utilisent diverses techniques :
\begin{itemize}
    \item Effacement sécurisé des données.
    \item Obfuscation et chiffrement.
    \item Stéganographie avancée.
\end{itemize}

Contremesures : déobfuscation, analyse comportementale, cryptanalyse forensique.

\section{Enjeux post-quantiques et cryptographie}

L’arrivée des ordinateurs quantiques remet en cause RSA et ECC (algorithmes de Shor et Grover). 
Contre-mesures : cryptographie post-quantique (Kyber, Dilithium), Zero-Knowledge Proofs, renforcement de la chaîne de custody.

Le Trilemme CRO reste central pour évaluer les compromis entre sécurité, confidentialité et valeur juridique.

\section{Cadre juridique et cas pratique Cameroun 2025}

\subsection{Cadre juridique}

\begin{itemize}
    \item États-Unis : Federal Rules of Evidence, CFAA.
    \item Europe : RGPD, eIDAS, Convention de Budapest.
    \item Afrique : Convention de Malabo (2014).
    \item Cameroun : Lois de 2010 et loi de 2024 renforçant l’investigation numérique.
\end{itemize}

\subsection{Cas pratique : CyberFinance Cameroun 2025}

\begin{enumerate}
    \item Détection : attaque par ransomware.
    \item Réponse initiale : confinement et préservation.
    \item Analyse technique : étude du ransomware.
    \item Collecte de preuves : norme ISO 27037 et protocole ZK-NR.
    \item Timeline \& attribution : identification des auteurs probables.
    \item Remédiation : défenses renforcées post-quantiques.
    \item Procédure judiciaire : dossier recevable devant les tribunaux camerounais.
\end{enumerate}

Ce cas illustre la transversalité : technique, droit, organisation.

\section*{Conclusion}

L’investigation numérique est une discipline incontournable à l’ère numérique. 
Elle repose sur la technique, l’éthique et le droit. 
L’investigateur doit :

\begin{itemize}
    \item Maîtriser des compétences pluridisciplinaires.
    \item Respecter des principes éthiques stricts.
    \item Intégrer les normes et cryptographies post-quantiques.
    \item Collaborer avec juristes et institutions.
\end{itemize}

L’avenir repose sur la capacité à anticiper les mutations technologiques, renforcer la confiance sociale et garantir l’opposabilité des preuves numériques dans un monde globalisé.

\end{document}
