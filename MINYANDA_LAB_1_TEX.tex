MINYANDA_RAPPORT_LAB_1
\documentclass[12pt,a4paper]{article}
\usepackage[utf8]{inputenc}
\usepackage[T1]{fontenc}
\usepackage{graphicx}
\usepackage{geometry}
\usepackage{setspace}
\usepackage{titlesec}
\usepackage{hyperref}

\geometry{margin=2.5cm}

\titleformat{\section}{\Large\bfseries}{\thesection}{1em}{}
\titleformat{\subsection}{\large\bfseries}{\thesubsection}{1em}{}

\begin{document}

% PAGE DE GARDE
\begin{center}

\includegraphics[width=0.35\textwidth]{img_0_13.png}\\[1cm]

{\Large \textbf{2025/2026}}\\[0.7cm]

{\large \textbf{RÉDIGÉ PAR : MINYANDA YOMGUI JOSÉ LOÏC}}\\[0.2cm]
{\large \textbf{MATRICULE : 22P071}}\\[0.2cm]
{\large \textbf{4e ANNÉE CYBERSÉCURITÉ ET INVESTIGATION NUMÉRIQUE}}\\[1.2cm]

{\Huge \textbf{RAPPORT LAB 1}}\\[1cm]

\includegraphics[width=0.35\textwidth]{img_0_14.png}

\end{center}

\newpage

% TABLE DES MATIÈRES
\begin{center}
\textbf{Contents}
\end{center}

\noindent INTRODUCTION .......................................................... 3\\
I. Conception ...................................................................... 4\\
1. Composition de l’architecture ........................................... 4\\
2. Schéma de l’architecture ................................................... 4\\
3. Plan d’adressage sur GNS3 ................................................. 4\\
II. Déploiement ...................................................................... 5\\
1. Création et configuration des machines virtuelles .................... 5\\
a) Machine Virtuelle Windows 10 .............................................. 5\\
b) Machine Virtuelle Linux (serveur web) ................................... 6\\
c) Création d’une application web ............................................ 6\\
2. Création de l’infrastructure ................................................ 6\\
a) Installation de GNS3 ......................................................... 6\\
b) Installation et configuration du pare-feu pfsense .................. 7\\
c) Adressage de la machine Windows et test du ping ............. 8\\
d) Adressage de la machine Linux et test du ping .................. 10\\
CONCLUSION ...................................................................... 12

\newpage

% INTRODUCTION
\section*{INTRODUCTION}

L'objectif du Lab 1 est de permettre aux étudiants de créer un environnement réseau complexe dans GNS3. 
Ils doivent configurer un réseau sécurisé incluant une machine Windows, une DMZ (zone démilitarisée) 
avec un serveur web sous Linux, un firewall, et un routeur. 

Ce Lab met en pratique les concepts de segmentation réseau et d'isolation des services critiques. 
Tout au long de ce document, nous allons détailler la mise en œuvre part à part de ce Lab jusqu’à 
son niveau fonctionnel.

\newpage

% I. CONCEPTION
\section{Conception}

\subsection{Composition de l’architecture}

Notre infrastructure réseau est composée de :

\begin{itemize}
\item Un routeur comme équipement de frontière
\item Une DMZ contenant un poste de travail (serveur Ubuntu) qui héberge une application web
\item Un poste de travail Windows 10 sur le réseau local contenant 2 Go de données
\end{itemize}

\subsection{Schéma de l’architecture}

\begin{center}
\includegraphics[width=0.8\textwidth]{img_3_64.png}
\end{center}

\subsection{Plan d'adressage sur GNS3}

\begin{center}
\includegraphics[width=0.8\textwidth]{img_4_66.png}
\end{center}

\newpage

% II. DÉPLOIEMENT
\section{Déploiement}

\subsection{Création et configuration des machines virtuelles}

Nous avons installé VMWare comme logiciel de virtualisation.

\subsubsection*{a) Machine Virtuelle Windows 10}

Nous avons créé une machine virtuelle avec les caractéristiques suivantes :  
DD : 20 Go ; RAM : 2 Go.

Nous avons copié et collé 2 Go de données sur le Bureau et dans le répertoire « Mes documents », 
et installé l’antivirus Avast.

\begin{center}
\includegraphics[width=0.8\textwidth]{img_5_68.png}
\end{center}

\subsubsection*{b) Machine Virtuelle Linux (serveur web)}

Machine Ubuntu : DD : 20 Go ; RAM : 2 Go.

\begin{center}
\includegraphics[width=0.8\textwidth]{img_6_70.png}
\end{center}

\subsubsection*{c) Création d’une application web}

Nous avons utilisé le CMS WordPress.  
Installation du serveur local XAMPP avec Apache et MySQL.  
Extraction de WordPress dans /other/opt/lampp, puis installation via \texttt{localhost/wordpress}.

\subsection{Création de l’infrastructure}

\subsubsection*{a) Installation de GNS3}

\begin{center}
\includegraphics[width=0.8\textwidth]{img_7_72.png}
\end{center}

\subsubsection*{b) Installation et configuration du pare-feu pfsense}

Nous avons choisi pfsense version 2.7.

\begin{center}
\includegraphics[width=0.8\textwidth]{img_8_74.png}
\end{center}

\newpage

\subsubsection*{c) Adressage de la machine Windows et test du ping}

Adresse : 192.168.1.20  
Interface LAN pfSense : 192.168.1.1

\begin{center}
\includegraphics[width=0.8\textwidth]{img_9_76.png}
\end{center}

\subsubsection*{d) Adressage de la machine Linux et test du ping}

\begin{center}
\includegraphics[width=0.8\textwidth]{img_10_78.png}
\end{center}

Lancement de XAMPP :

\begin{center}
\includegraphics[width=0.8\textwidth]{img_10_79.png}
\end{center}

\newpage

% CONCLUSION
\section*{CONCLUSION}

Nous voici arrivés au terme de notre Lab.  
Le ping passe entre les différentes machines et l’application web est accessible depuis la VM Windows.

\vspace{1cm}

\textbf{MATRICULE : 22P071}\\
\textbf{RÉDIGÉ PAR : MINYANDA YOMGUI JOSÉ LOÏC}

\end{document}
